\chapter{Deep learning theory}\label{ch:ml}
\section{Introduction}

The research question being explored in this thesis is to what degree we can extract compressed information about physical events from the AT-TPC experiment using modern machine learning methods. To achieve this we employ the DRAW algorithm (\cite{Gregor2015}) and variations of a traditional autoencoder. The DRAW algorithm is built of neural network components in a joint architecture comprised of a variational autoencoder wrapped in a set of long term short term memory cells. Each  of the components are discussed in their own sections starting with the neural network in  section \ref{sec:ANN} then followed by autoencoders in section \ref{sec:autoencoder} and finally recurrent neural networks in \ref{sec:rnn}. 

To arrive at the DRAW network we need to introduce the optimization of the log likelihood function using a binary cross-entropy cost function. In it's simplest form this optimization problem occurs in the formulation of the logistic regression mode introduced in section \ref{sec:LogReg}. As part of the derivation of the variational autoencoder cost the same optimization problem of the log likelihood will be applied. Likewise we introduce gradient descent methods and regularization, crucial components of modern machine learning, in the familiar framework of linear regression in section \ref{sec:LinReg}. 

We hypothesize that this compressed information can be used to linearly separate events in classes, possibly using relatively small amounts of data. Furthermore we hypothesize that we can construct an implicit clustering based on emergent structures in the latent space. In the experiment at hand we hope to separate events with proton or a carbon as the reaction output. 